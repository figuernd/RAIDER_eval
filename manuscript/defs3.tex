\documentclass{article}
\usepackage{longtable}
\usepackage{ulem}
\pagestyle{myheadings}
\usepackage{ifthen}
\usepackage{color}

\setlength{\parindent}{0pt}
\setlength{\parskip}{1ex plus 0.5ex minus 0.2ex}

\newcommand{\und}{\underline}

\newcommand{\red}[1]{{\color{red}#1}}
\newcommand{\blue}[1]{{\color{blue}#1}}
\newcommand{\cyan}[1]{{\color{cyan}#1}}
\newcommand{\magenta}[1]{{\color{magenta}#1}}

\newtheorem{theorem}{Theorem}
\newtheorem{notation}{Notation}
\newtheorem{definition}{Definition}
\newtheorem{observation}{Observation}
\newtheorem{lemma}{Lemma}

\begin{document}

\section{REFDs}

\begin{definition}
  A \underline{sequence descriptor} (sd) is a string over $\Sigma=\{A,C,G,T,*\}$.
\end{definition}

We will allow any normal string operator to apply to sequence descriptors.  In specific, for an s.d. $r$:
\begin{itemize}
\item $r_{i,i+l}$, or $r[i:i+l]$, is the length $l$ sub-sd (substring) of $r$ starting at $i$.  (Python-like notation.)
\item $|r|$ is the number of characters in $r$.
\item $||r||$ is the number of non-* characters in $r$.
\item We will use $l$-mer to denote a length $l$ sd.
\end{itemize}

\begin{definition}
  If $r_1$ and $r_2$ are sds of the same length:
\begin{enumerate}
\item $r_1$ and $r_2$ are \und{consistent} if, for any position $i$ in
  which they differ, one of them has a $*$ in that position.
\item $r_2$ \und{generalizes} $r_1$ (and $r_1$ is \und{generlized} by $r_2$),
  if, for any position in which they are different, $r_2$ has a $*$.
  \end{enumerate}
\end{definition}

\begin{definition}
  The frequency of an sd $r$ in genome sequence $G$ is the number of
  $i$ such that $G_{i:i+|r|}$ is compatible with $r$.  
\end{definition}
{\bf Note:} Debated whether this should be ``compatible'' or ``generalized'' -- then realized that since $G$ is a string (as opposed to an sd) they are equivilent.


\begin{notation}
Let $r_1$ and $r_2$ be sds.  We say:
\begin{itemize}
\item $r_1 < r_2$ if $r_1$ is generalized by $r_2$.  Extended to
  $\leq$, $\geq$, and $>$ as normal. 
\item $r_1 || r_2$ ($r_1$ is ``parallel'' to $r_2$) if $r_1$ and $r_2$ are
  compatible, but neither generalizes the other.
\item $freq_G(r)$ is number of substrings in $G$ that are compatable with $r$.
\end{itemize}
\end{notation}

{\bf Comment:} This forms a lattice over the set of all
$l$-length sds.  I have no idea if this is important/useful.

\section{REFDs and Spaced Seeds}

\begin{definition}
Given a spaced seed of length $l$ and an $l$-mer $r$, we
\und{generlize} $r$
w.r.t $s$ by replacing $r_i$ with a * for all $i$ such that $s_i = 0$.
\end{definition}
Example: We generalize $AAAAAA$ w.r.t. $110101$ to $AA*A*A$.

\begin{definition}
  Let $s$ be a spaced seed of length $l$.
 
  \begin{itemize}
  \item We say $s$ and an $l$-mer $r$ are \und{consistent} if, when $r_i = *$, $s_i=0$.
  \item We say seed $s$ \und{hits} sd $r$ at position $i$ such that
    $s$ is consistent with $r_{i:i+|s|}$ ($i \leq |r| - |s|$).
  \item We say seed $s$ \und{covers} sd $r$ if, for every $j$, $0 \leq
    j < |r|$, there is an $i$ such that $s$ hits $r$ at $i$ and covers
    position $j$ in doing so.
  \end{itemize}
\end{definition}

Example: For $s=11011$ and $r=AAAAA*TTTTT$, $s$ covers $r$, hitting at
$\{0,3,6\}$.  $s$ does not cover $AAAAA**TTTTT$, as there is no
position at which $s$ hits and extends over $i=5$.

\begin{definition}
  Let $r$ be an sd and $s$ be a seed that covers $r$.
\begin{itemize}
\item The \und{decompostion} of $r$ w.r.t $s$ is set of all $i$ such
  that $s$ hits $r$ at position $i$.
\item The \und{seqeunce decomposition} of $r$ w.r.t. $s$ is the set
  of all sds $r_{i:i+|s|}$ for each $i$ in the decomposition.
\item The \und{generalized sequence decomposition} of $r$ w.r.t. $s$
  is the set of generalizations of $w$ for all $w$ in the sequence decomposition.
\end{itemize}
\end{definition}
  
Examples: For $s=11011$ and $r = TAAAAA*CCCCCC$, the decomposition is
$\{1,2,5,8,9\}$, the sequence decomposition is $\{TAAAA, AAAAA, AA*CC,
CCCCC\}$ (note the two sequences a 8 and 9 are the same, so only
included once), and the generlized sequence decomposition is $\{TA*AA,
AA*AA, AA*CC, CC*CC\}$. 
    
\begin{notation}
Notation for the concepts from this section:
  \begin{itemize}
  \item Let $\gamma_s(r)$ denote the generalization of $r$ w.r.t. $s$
  (undefined if $|r| \neq |s|$).
  \item Let $\delta_s(r)$ denote the decomposition of $r$ w.r.t. $s$.
  \item Let $\Delta_s(r)$ denote the sequence decomposition of $r$ w.r.t. to
  $s$.
  \item Let $\Gamma_s(r)$ denote the generalized seqeunce decomposition of
  $r$ w.r.t. $s$.
  \end{itemize}
\end{notation}

{\bf Note:} I messed around with other decompositions (subsets of the
decomposition, minimal decompositions, right-most decompositions), but
this is the only one that seemed to be useful so far.

\section{Elementary repeats}

\begin{definition}
  \label{ERdef}
  Given a fixed seed $s$, genome $G$, and frequency $f$, an sd $r$ is
  a \underline{repeat element descriptor} (red) if the following
  holds true:
  \begin{enumerate}
  \item $s$ covers $r$.
  \item $freq_G(r) \geq f$.
  \item $freq_G(w) = freq_G(r)$ for all $w \in \Gamma_s(r)$.
  \item There does not exist an $r'$ with a substring $w$ such that
    $r \leq w$ and $r'$ meets conditions 1-3.
  \end{enumerate}
\end{definition}
{\bf Questions:} On (4): what if there exists a substring $w$ of $r'$
such that $r || w$?  What if $w$ is not a proper substring?  (That is,
$r = w$.)

\begin{definition}
  A {\it repeat element family} is a set $S$ of genome coordinates
  such that there exists an sd $r$ where:
\begin{itemize}
\item $r$ generlizes the string $G_{i:i+|r|}$ for every $i \inS4.
\item $r$ does not generalize any other substring of $G$.
\end{definition}
Notice that there can be multiple such $r$ for a given set.  Any two
such $r$ for the same $S$ will be compatible, but they don't
necessarily generlize each other.


\section{Equivilance proof}

We want to prove that our definition, when using a seed with no
spaces, is equivilent to the Z\&L definition.  Recall that in Nate's
there was an alternate definition proved equivilent to Z\&L.

\begin{definition}
\label{NateDef}
Given a genomic sequence $G$ and integers $f$ and $l$, a seqeunce $r$
is an {\it elementary repeat} if we can write $r = x_0 \circ x_2 \circ
\cdots \circ x_k$ where $|x_i|=l$, $k = |r|-|l|$, and:
\begin{enumerate}
\item $k \geq 0$.  (There is at least one $l$-mer substring).
\item $freq_G(r) \geq f$
\item $freq_G(x_i) = freq_G(r)$
\item There does not exist an $y$, $|y|=l$ such that either $y \circ
  r$ or $r \circ y$ meets conditions 1-3.
\end{enumerate}
\end{definition}
Recall that $x \circ y$ is defined only when $x$ has an $|x|-1$ length suffix
that is a prefix of $y$, and represents their ``merging''.
(e.g. $ACCC \circ CCCG = ACCCG$.)


\begin{theorem}
  Consider a genome $G$, an integer $f$, and a seed $s$ such that $s$
  has no zero chartacters.  $r$ is an elementary
  repeat descriptor for $s$ (by Definition~\ref{ERdef}) if and only if $r$
  is an elementary repeat for $l=|s|$ (by
  Definition~\ref{NateDef})).
\end{theorem}

{\bf Proof:}

($\Leftarrow$) First we observe that if $s$ has no zeros, and covers
$r$, then $r$ has no * characters.  From this it follows that
$\Gamma_s(r) = \{r_0,r_1,\cdots,r_{k}\}$ where $|r_i|=l$ and $k=r-l$.
That is, every $l$-mer contained with $r$.  From this it follows that $r = r_0 \circ
r_1 \circ \cdots \circ r_k$.


Now we prove each of the four conditions of Nate's definiciton:
\begin{enumerate}
  \item Since $r$ is covered by $s$, $|r| \geq |s|$, so $k =
    |r|-|s|+1 > 0$.
  \item Since $r$ contains no * symbols, the only string it
    is compatible with is itself.  Meaning it has the same frequency by
    either definition (sd or string).  Hence (2) of Nate's
    definition follows directly from (2) of our definition.
  \item Follows immediately from the fact that every $r_i$is a
    member of $\Delta_G(r)$, and hence must have the right frequency.
   \item Argue by contradiction: suppose there were such a $y$ (assume
     w.l.g. it follow $r$), and let $r' = r \circ x$.  Since Nate's
     definition tells us $r'$ matches a string around every occurance
     of a match to $r$, this means $r'$ is (trivially) compatable to a
     substring around every occurance of a match to $r'$ -- so $r'$
     could not have been a repeat element descriptor.
\end{enumerate}

($\Rightarrow$) This is the same argument backwards, and not worth
writing out.

\section{Basic Lemmas}

\begin{definition}
  Given a repetitive element descriptor $r$ for seed $s$ of length
  $l$, we say a length $l$ s.d. $w$ belongs to $r$ is $w$ is
  consistent with so $w \in 
  \Gamma_s(r)$.  (That is: if there is a a string in the generlized
  decomposition of $r$ that is consistent with $r$.
\end{definition}


Using this definition of {\it belongs}, all of our lemmas and
observations still holds:
\begin{lemma}
  An $l$-mer belongs to at most one elementary repeat family.
\end{lemma}
If the $l$-mer belongs, then it subject to condition (3) of the
definition, which would be violated if the $l$-mer belonged to two
different families.

\begin{observation}
  Any $l$-mer that occurs $\geq f$ times either is an elementary
  repeat descriptor or belongs to one.
\end{observation}
If the $l$-mer occurs at least $f$ times is automatically meets
criteria 1-3.  So the only thing that could stop it from being an
elementary repeat is (4) -- meaning its contained within one.

\begin{lemma}
  Let $r$ be a sd that maximially matches $G$ in two locations.  Than
  $r$ may be an elementary repeat descriptior, or may contain an
  elementary repeat descriptor -- but it is not contained within
  another elementary repeat descriptor.  (That is, (4) must hold.)
\end{lemma}
Any elementary repeat descriptor containing it would have to hit at
both location, which we just said isn't possible.

{\bf COMMENT:} HMM...I'm worried about this one.  What if we extneded
by a whole se1d length?  Could that matter?

\begin{lemma}
  If two $l$-mers belong to the same elementary repeat
  descriptor, then they have the same frequency.
  
\end{lemma}
Follows directly from the definition.

From here is should be very easy to prove that Carly's algorithm
returns an elementary repeat descriptors for each family.

\section{Variations}

Suppose $r_1$ and $r_2$ are repeat element descriptors for the same
set of instances.  What are their potential relationships?

Let $s=11011$ and $G = \cdots AAAAACGGGGG \cdots AAAAATGGGGG\cdots$.
Then possible repeat element descriptors are $11111*11111$,
$11*11*111111$, $11111*11*11$, and $11*11*11*11$.  So we can have $r_1
< r_2$ or $r_1 || r_2$.

For $11011$ and $G = \cdots ACCCCCGTTTTTA\cdots ACCCCCATTTTTA$, we can
have $AC*CCC*TT*TTA$ or $ACC*CC*TTT*TA$, among others.

\subsection{Carly's output}

\begin{definition}
Given a repeat element family $S$ and a descriptor $r$, we say:
\begin{itemize}
\item We say position $i$ is variable in $S$ if there are two elements
  of $S$ whose sequences vary at positin $i$.
\item Position $i$ of $r$ is \und{tight} if either $r_i \neq *$ or
  $S$ variable at position $i$.  (That is, $r$ must have a *.)
\item We say $r$ is tight if $r_i$ is tight for all $0 \leq i < |r|$.
\end{itemize}
\end{definition}

Facts about Carly's output:
\begin{itemize}
\item If $S$ is invarient in all positions, then the results $r$ has
  no * characters.  For every $i$, $s$ hits at either $i$ or
  $i-|s|+1$, meaning that there is a $w \in \Gamme_s(r)$ that starts
  (or ends) at position $i$.  Since a seed starts (ends) with a 1,
  there canot be a * at this position.
\item The output descriptor is not necessarily tight.  For example:
  $s=11011$, where $G$ contains $AAAAACGGGGG$ and $AAAAATGGGGG$.  Then
  $r = AA*AA*AA*AA$, hence is not tight at position 2.
\end{itemize}




  
    
  




\end{document}
