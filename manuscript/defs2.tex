\documentclass{article}
\usepackage{longtable}
\usepackage{ulem}
\pagestyle{myheadings}
\usepackage{ifthen}
\usepackage{color}

\setlength{\parindent}{0pt}
\setlength{\parskip}{1ex plus 0.5ex minus 0.2ex}

\newcommand{\red}[1]{{\color{red}#1}}
\newcommand{\blue}[1]{{\color{blue}#1}}
\newcommand{\cyan}[1]{{\color{cyan}#1}}
\newcommand{\magenta}[1]{{\color{magenta}#1}}

\newtheorem{definition}{Definition}
\newtheorem{observation}{Observation}
\newtheorem{lemma}{Lemma}

\begin{document}

\section{Definitions}

\subsection{REFDs}
In RAIDER, everything is defined in terms of $l$-mers.  hHence we need
to generlize slightly to allow the ambiguity character, so we will
create a description of a {\it repeat element family discriptior},
with an $l$-mer being a special case of that.

\begin{definition}
A \underline{repeat element family discriptor} (refd) is a sequence over
$\Sigma=\{A,C,G,T,*\}$, where $*$ denotes unspecified bases. 
\end{definition}

{\bf Note:} We will refer to sub-refds as consistent with the concept
of a substring, and denote then as consistent with Python.  (That is,
if $r =  AAAC*CGG'$, then the sub-refd $r_{3:6}$, or $r[3:6]$, is the refd
$C*C$).  Generally, we will define all standard string operators on
refds unless otherwise noted.  (e.g. $\cdot$ will be the concatenation
operator, $|r|$ will denote total length, etc...)

\begin{definition}
Given an refd, $|r|$ denotes the length of the string and $||r||$
denotes the number of non-$*$ characters in the string.
\end{definition}
For $r = A*A$, $|r| = 3$ and $||r|| = 2$.

\begin{definition}
An $l$-mer is an refd of length $l$.
\end{definition}

\begin{definition}
Let $r$ and $r'$ be two $l$-mers.
\begin{itemize}
\item $r$ and $r'$ are \underline{consistent} if,
  for any position $i$ in which they differ, one of them contains a
  $*$.  ($AAC*TT$ and $AA*GTT$ are consistent; $AAC*TT$ and $AAGGTT$
  are not.)
\item We will say $r$ \underline{generalizes} $r'$ if, wherever they differ,
  $r$ has a $*$.  Denote this as $r \vdash r'$.  ($AAC*TT \vdash AACGTT$, 
  $AAC*TT \not\vdash AA*CTT$.)
\item The \underline{merge} of $r$ and $r'$, denoted $r \oplus r'$, is
  the $l$-mer such that $s_i = r_i$ when $r_i=r'_i$,
  and $*$ otherwise. ($AAC*TT \oplus AA*GTT = AA**TT$.) 
\item The \underline{tight merge} of $r$ and $r'$, denoted $r \otimes r'$,
  is $r \oplus r'$ if $r$ and $r'$ are consistent, and undefined
  otherwise. ($AAC*TT \otimes AA*GTT = AA**GT$, but $AAC*TT \otimes AAAA*CGT$ is undefined.)
\item The \underline{compression} of consistent refds $r$ and $r'$, denoted $r\ominus r'$, 
  is the string $s$ where $s_i$ is: $*$ if both $r_i$ and $r'_i$ are
  $*$, other the non-$*$ character at $r_i$ or $r'_i$.  If $r$ is and
  $r'$ are not consistent, $r \ominus r'$ is undefined.  ($AA**C
  \ominus A**CC = AA*CC$.
\end{itemize}
\end{definition}

{\bf Comments:}
\begin{itemize}
\item $r \oplus r'$ is the string that finds the $s$ over all
$l$-mers $s$ that generalizes both $r$ and $r'$ which maximizes $||s||$.
\item $r \ominus r'$ is the string that finds the $s$ over all
  $l$-mers $s$ that can be gerlized t both $r$ and $r'$ which
  maximizes $||s||$.
\item It might makes sense to say $r' < r$ if $r$ generlizes $r$ --
  which would then indce a lattice.  I have no idea if this is
  actually interesting / useful, and am ignoring it for now.
\end{itemize}


\begin{definition}
Let $x$ and $y$ be REFDs. Define $x \circ y$ as follows.  Let $i$ be
the largest $i$ such that $x_{|x|-i:}$ and $y_{:i}$ are consistent, let
$b = x_{|x|-i:} \ominus y_{:i}$, let $a = x_{:|x|-i}$ and $c = y_{i:}$.  
$x \circ y = a \cdot b \cdot c$.
\end{definition}
{\bf Examples:}
\begin{itemize}
\item $AAACCC \circ C*CTTT = AACCC$.
\item $AAACCC \circ CTCTTT = AAATCCTCTTT$.
\item $AAACC* \circ CC*TTT = AAACC*TTT$.
\end{itemize}


\subsection{REFDs and Spaced Seeds}


\begin{definition}
Given an $l$-mer $r$ and a spaced seed $s$ of length $l$, The seeded
$l$-mer, denoted $\sigma_s(r)$, is created by removing all bases at
positions corresponding to 0s.
\end{definition}
Example: $\sigma_{11011}(AACGG) = AAGG$. 

\begin{definition}
Given an $l$-mer $r$ and a spaced seed $s$ of length $l$, the
generlized $l$-mer (w.r.t. $s$), or $\gamma_s(r)$, is the refd created by replacing
all basses corresponding to 0s with * characters.
\end{definition}
Example: $\sigma_{110101}(AACGTG) = AA*G*G$.

\begin{definition}
Given a spaced seed $s$ of length $l$: 
\begin{itemize}
\item $s$ is {\it consistent} with an $l$-mer $r$ if, for all
$i$ such that $r_i = *$, $s_i = 0$.  
\item A spaced seed $s$ hits an refd $r$ at position $i$ if 
  $i <|r|-|s|$ and $s$ is consistent with $r_{i:i+|s|}$.
\end{itemize}
\end{definition}

\begin{definition} 
A spaced seed $s$ of length $l$ {\underline covers} and refd $r$ is, for every $j$,
there is an $i \leq j$ such that $s$ is consistent with the sub-refd
$r_{i:i+|l|}$.
\end{definition}
{\bf Examples:} Let $s = 11011$.
\begin{itemize}
\item $r = AAAAA*AA$. $s$ covers $r$ as we can place $s$ $i=0$ and
  $i=3$ and, between them, cover every character in $r$ without
  conclict between a 1 and a $*$.
\item $r = AAA*AA$.  $s$ does not cover $r$.  $s$ only hist at $i=1$,
  so there is no way to cover base 0.
\item $r = AAAAAAAAAA$.  $s$ covers $r$, since $s$ hits at every base.
\end{itemize}

\subsection{REFDs and Sequence Sets}
Looks like we will need to define the relationship between refds and
sets of sequences to get the elementary repeat sequence.

Assume a fixed genome $G$.

\begin{definition}
Let $r$ be an refd of length $l$ and let $S$ be a set of sequences,
each of length $l$.
\begin{itemize}
\item We say $r$ is \underline{consistent} with $S$ if $r$ is
  consistent with every sequence in $S$.
\item We say $r$ is \underline{tightly consistent} with $S$ if $r$ is
  consistent with $s$ and, for every $i$ such that $r_i=*$, there
  exists sequences $x,y \in S$ such that $x_i \neq y_i$.
\end{itemize}
\end{definition}
That is: $r$ is tightly consistent with the set $S$ is $r$ only has
$*$ characters in those places it must.  (I thought this was going to
be necessary at one point, but am now doubting it.)




\subsection{Elementary Repeats}

\begin{definition}
Let $s$ be a fixed seed with length $l$, $f$ be a fixed integer, and $G$ be a fixed
genomic sequence.  An
\underline{Elementary Repeat Family} is a tuple $(r,S)$ where: 
\begin{enumerate}
\item $r$ is an refd such that $s$ covers $r$. 
\item $S$ is a set of genomic coordinates such that $|S| \geq f$.  Say
  that the sub-refds induced by $S$ are the substrings of $G$ starting
  at cordinates of $S$ and extended $|r|$ bases.
\item For every $i \in S$, $r$ is consistent with the string
  $G_{i:i+|r|}$.
\item Minimality condition.  Possibilities (I'm not sure which is
  right):
  \begin{itemize}
  \item There is no sub-refd $r'$ of $r$ that is covered by $s$ and is
     consistent some sequence $x$ that is sits outside the sequences
      induced by $S$.
  \item There is no $l$-mer $r'$ contained in $r$ that is consistence
    with $s$, such that $\sigma_s(r)'$ occurs anywhere in $G$ outside
    of the sequences induced by $S$.
 \end{itemize}
\item There is no refd $r'$ containing $r$ such that every sequence
  induced by $S$ is contained by some sequence consisted with $r'$.
\end{enumerate}
\end{definition}

{\bf Comments:} For the last two points, I'm really not sure if the
substring relation is the correct relation between $r$ and $r'$.


\subsection{Lemmas}

Right now, I'm trying trying to think ofpossibly interesting
statements and prove them.

Assume a fixed seed $s$ of length $l$ and a fixed genome $G$.






\end{document}
