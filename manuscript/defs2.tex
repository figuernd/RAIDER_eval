\documentclass{article}
\usepackage{longtable}
\usepackage{ulem}
\pagestyle{myheadings}
\usepackage{ifthen}
\usepackage{color}

\setlength{\parindent}{0pt}
\setlength{\parskip}{1ex plus 0.5ex minus 0.2ex}

\newcommand{\red}[1]{{\color{red}#1}}
\newcommand{\blue}[1]{{\color{blue}#1}}
\newcommand{\cyan}[1]{{\color{cyan}#1}}
\newcommand{\magenta}[1]{{\color{magenta}#1}}

\newtheorem{definition}{Definition}
\newtheorem{observation}{Observation}
\newtheorem{lemma}{Lemma}
\newtheorem{theorem}{Theorem}

\begin{document}

\section{Definitions}

\subsection{REFDs}
In RAIDER, everything is defined in terms of $l$-mers.  Hence we need
to generalize slightly to allow the ambiguity character, so we will
create a description of a {\it repeat element family description},
with an $l$-mer being a special case of that.

\begin{definition}
A \underline{repeat element family discriptor} (refd) is a sequence over
$\Sigma=\{A,C,G,T,*\}$ (where $*$ denotes unspecified bases).
\end{definition}

{\bf Note:} We will refer to sub-refds as consistent with the concept
of a substring, and denote then as consistent with Python.  (That is,
if $r =  AAAC*CGG$, then the sub-refd $r_{3:6}$, or $r[3:6]$, is the refd
$C*C$).  Generally, we will define all standard string operators on
refds unless otherwise noted.  (e.g. $\cdot$ will be the concatenation
operator, $|r|$ will denote total length, etc...)

\begin{definition}
Given an refd, $|r|$ denotes the length of the string and $||r||$
denotes the number of non-$*$ characters in the string.
\end{definition}
For $r = A*A$, $|r| = 3$ and $||r|| = 2$.

\begin{definition}
An $l$-mer is an refd of length $l$.
\end{definition}

\begin{definition}
Let $r$ and $r'$ be two $l$-mers.
\begin{itemize}
\item $r$ and $r'$ are \underline{consistent} if,
  for any position $i$ in which they differ, one of them contains a
  $*$.  ($AAC*TT$ and $AA*GTT$ are consistent; $AAC*TT$ and $AAGGTT$
  are not.)
\item We will say $r$ \underline{generalizes} $r'$ if, wherever they differ,
  $r$ has a $*$.  Denote this as $r \vdash r'$.  ($AAC*TT \vdash AACGTT$, 
  $AAC*TT \not\vdash AA*CTT$.)
\item The \underline{merge} of $r$ and $r'$, denoted $r \oplus r'$, is
  the $l$-mer such that $s_i = r_i$ when $r_i=r'_i$,
  and $*$ otherwise. ($AAC*TT \oplus AA*GTT = AA**TT$.) 
\item The \underline{tight merge} of $r$ and $r'$, denoted $r \otimes r'$,
  is $r \oplus r'$ if $r$ and $r'$ are consistent, and undefined
  otherwise. ($AAC*TT \otimes AA*GTT = AA**GT$, but $AAC*TT \otimes AAAA*CGT$ is undefined.)
\item The \underline{compression} of consistent refds $r$ and $r'$, denoted $r\ominus r'$, 
  is the string $s$ where $s_i$ is: $*$ if both $r_i$ and $r'_i$ are
  $*$, other the non-$*$ character at $r_i$ or $r'_i$.  If $r$ is and
  $r'$ are not consistent, $r \ominus r'$ is undefined.  ($AA**C
  \ominus A**CC = AA*CC$.  But $AA**C \ominus AT**C$ is undefined
  (because of the disagreement in the second position).
\end{itemize}
\end{definition}

{\bf Comments:}
\begin{itemize}
\item $r \oplus r'$ is the string that finds the $s$ over all
$l$-mers $s$ that generalizes both $r$ and $r'$ which maximizes $||s||$.
\item $r \ominus r'$ is the string that finds the $s$ over all
  $l$-mers $s$ that can be gerlized t both $r$ and $r'$ which
  maximizes $||s||$.
\item It might makes sense to say $r' < r$ if $r$ generlizes $r$ --
  which would then indce a lattice.  I have no idea if this is
  actually interesting / useful, and am ignoring it for now.
\end{itemize}


\begin{definition}
Let $x$ and $y$ be REFDs. Define $x \circ y$ as follows.  Let $i$ be
the largest $i$ such that $x_{|x|-i:}$ and $y_{:i}$ are consistent, let
$b = x_{|x|-i:} \ominus y_{:i}$, let $a = x_{:|x|-i}$ and $c = y_{i:}$.  
$x \circ y = a \cdot b \cdot c$.
\end{definition}
{\bf Examples:}
\begin{itemize}
\item $AAACCC \circ C*CTTT = AAACCCTTT$.
\item $AAACCC \circ CTCTTT = AAATCCTCTTT$.
\item $AAACC* \circ CC*TTT = AAACC*TTT$.
\end{itemize}

\begin{definition}
Given aan refd and a sequence $G$, let $freq_G(r)$ be the number of
positions $i$ at which $r$ generalizes $G_{i:i+|r|}$.
\end{definition}

{\bf Example:} Let $R=AA*AA$ and $G=AAAAACCAAGAAC$.  Then $freq_G(r)
= 2$.  



\subsection{REFDs and Spaced Seeds}


\begin{definition}
Given an $l$-mer $r$ and a spaced seed $s$ of length $l$, the seeded
$l$-mer, denoted $\sigma_s(r)$, is created by removing all bases at
positions corresponding to 0s.
\end{definition}
Example: $\sigma_{11011}(AACGG) = AAGG$. 

\begin{definition}
\label{generalize}
Given an $l$-mer $r$ and a spaced seed $s$ of length $l$, the
generlized $l$-mer (w.r.t. $s$), or $\gamma_s(r)$, is the refd created by replacing
all basses corresponding to 0s with * characters.
\end{definition}
Example: $\gamma_{110101}(AACGTG) = AA*G*G$.

\begin{definition}
Given a spaced seed $s$ of length $l$: 
\begin{itemize}
\item $s$ is {\it consistent} with an $l$-mer $r$ if, for all
$i$ such that $r_i = *$, $s_i = 0$.  
\item A spaced seed $s$ hits an refd $r$ at position $i$ if 
  $i <|r|-|s|$ and $s$ is consistent with $r_{i:i+|s|}$.
\end{itemize}
\end{definition}

\begin{definition} 
A spaced seed $s$ of length $l$ \underline{covers} and refd $r$ is, for every $j$,
there is an $i \leq j$ such that $s$ is consistent with the sub-refd
$r_{i:i+|l|}$.
\end{definition}
{\bf Examples:} Let $s = 11011$.
\begin{itemize}
\item $r = AAAAA*AA$. $s$ covers $r$ as we can place $s$ at $i=0$ and
  $i=3$ and, between them, cover every character in $r$ without
  concflict between a 1 and a $*$.
\item $r = AAA*AA$.  $s$ does not cover $r$.  $s$ only hits at $i=1$,
  so there is no way to cover base 0.
\item $r = AAAAAAAAAA$.  $s$ covers $r$, since $s$ hits at every base.
\end{itemize}


\subsection{REFDs and Sequence Sets}
Looks like we will need to define the relationship between refds and
sets of sequences to get the elementary repeat sequence.

Assume a fixed genome $G$.

\begin{definition}
Let $r$ be an refd of length $l$ and let $S$ be a set of sequences,
each of length $l$.
\begin{itemize}
\item We say $r$ is \underline{consistent} with $S$ if $r$ is
  consistent with every sequence in $S$.
\item We say $r$ is \underline{tightly consistent} with $S$ if $r$ is
  consistent with $s$ and, for every $i$ such that $r_i=*$, there
  exists sequences $x,y \in S$ such that $x_i \neq y_i$.
\end{itemize}
\end{definition}
That is: $r$ is tightly consistent with the set $S$ is $r$ only has
$*$ characters in those places it must.  (I thought this was going to
be necessary at one point, but am now doubting it.)

\subsection{REFD Decomposition}

Given an REDF $r$ and a seed $s$ of length $l$ ($l \leq |r|$), we want
to decompose $r$ into a set of overlapping $l$-mers.  These are the
$l$-mers that cannot be repeated outside of the repeat.

For example: if we have $G = \ldots AAAAACTTTTT\ldots AAAAAGTTTTT
\ldots$ (notice: 6 As), with the seed $1101$, we get the refd $AAAAAA*TT*TT$, which
deposes into $AA*AA$, $AA*TT$, and $TT*T$.  If any string compatible
with any of these three occurs anywhere else, it breaks the family.
But if the string $AA*AC$ occurs somewhere else, it does not -- thats
irrelevant.  (In this example the minimal decomposition is unique -- I'm not
sure that this is always the case, or what multiple decompositions
mean.)

\begin{definition} 
Given a spaced seed $s$ of length $l$ and a an refd $r$ that is
covered by $s$, a \underline{decomposition} of $r$ w.r.t. $s$ is a set of coordinates $i$, $0 \leq i < |r|-l$, such
that (a) for each $i \in S$, $s$ hits $r$ at $i$, and (b) $r$ is
covered by $s$ using only those positions from $S$.
\end{definition}

Examples:
\begin{itemize}
\item $s=11011$, $r=AAAAA*CCCCC$, then the only decomposition is
  ${0,3,6}$.
\item $s=11011$, $r=AAAAAA*CCCCC$, then ${0,1,4,7}$ and ${0,4,7}$ are
  both decompositions.
\end{itemize}

\begin{definition}
A decomposition $S$ of $r$ with respect to $s$ is \underline{minimal}
if no element of $S$ can be removed without breaking the decomposition
properties. 
\end{definition}
In the above, the first set of the second example is not minimal. 

\begin{definition}
A decomposition $S$ of $r$ w.r.t. $s$ is \underline{maximal} it
countains every $i$ such that $s$ hits $r$ at $i$.
\end{definition}
(Clearly this is unique.)

Eample: 
\begin{itemize}
\item If $s=11011$ and $r=AAAAAA*CCCCCC$, then $D=\{0,1,4,7,8\}$.
\item If $s$ has no 0s, then $D = \{i \; : \l 0 \leq i \leq
  |r|-|s|\}$.
\end{itemize}

\begin{definition}
Given an refd $r$, a length $l$ covering seed $s$, and a decomposition $D$, the
\underline{sequence decomposion} $Q = \{\gamma(r_{i:i+l}) \; : \; i \in D}$ 
\end{definition}

{\bf Examples:}
\begin{itemize}
\item $s=11011$, $r=AAAAA*CCCCC$, the only minimal decomposition is
  $D=\{0,4,6\}$, and the corresponding sequence decomposition is
  $\{AA*AA, AA*CC, CC*CC\}$.
\item $s=11011$, $r=AAAAATCCCCC$, there are five different minimal
  decompositions.  For $\{0,1,6\}$ the sequence decomposition would be
  $AA*AA, AA*AT$, and $TT*TT$.
\end{itemize}

\begin{definition}
The {\it representitive sequence} of a decomposition is
created by compressing the overlapping portions of the elements of the
decompresion.
\end{definition}

{\bf Examples:}
\begin{itemize}
\item $s=11011$, $r=AAAAA*CCCCC$, $D={0, 3, 6}$,
  $Q=\{AA*AA,AA*CC,CC*CC$}.  The representative sequence would be
  $\{AA*AA*CC*CC\}$.
\item $s=11011$, $r=AAAAATCCCCC$, $D={0,2,6}$, $Q=\{AA*AA, AA*TC, CC*CC\}$.  
 The representative sequence would be $\{AAAAATCC*CC\}$.  We have to
 compress the overlap at $r=2:5$, which endsup removing both *
 symbols.
\end{itemize}

{\bf TO CONSIDER: What is the relationship between $r$ and the
  representative sequence of one of its decompostion?  Always a
  generlization of $r$?}


\subsection{Elementary Repeats}

\begin{definition}
For a ferquenct $f$, a genome $G$ and seed $s$, an refd $r$ is an \undetline{elementary
  repeat} if the following holds.  
\begin{enumerate}
\item $s$ covers $r$.
\item $freq_G(r) \geq f$.
\item For the maximimal sequence decomposition $Q$, 
  $freq_G(q) = freq_G(r)$ for all $q \in Q$.
\item There does not exist an $r'$ that is covered by $s$ such that
  (1) $r$ is compatable(2) with a sub-refd of $r'$, and (2) $r'$ meets
  conditions (1)-(3) of this definition.
\end{enumerate}
\end{definition}


\begin{theorm}
If $s$ is a contiguous spaced seed (that is, no 0s) and $r$ is a
repeat element w.r.t to $s$, the $r$ is a repeat element by Definition
5 of Nate's thesis.
\end{theorem}
{\bf Proof:} Let $s$ be a spaces seed of length $L$ with no 0 symbols,
and $r$ be an elementary repeat.  Note that $r$ has no $*$
symbols, as otherwise $s$ could not cover it.  Now consider the
conditions for Definition 5:
\begin{itemize}
\item The repeat element is at least length $L$: since $r$ is covered
  by seed $s$ of length $L$, this is clearly true.\footnote{There
    appears to be a mistake in Nate's thesis with condition 1 -- this
    is what it was supposed to mean.}
\item $freq_G(r) \geq f$: Follows directly from (2).
\item Every substring of $r$ has the same frequence as $r$: Since $s$
  has no 0s, $Q$ is the set of all substrings of $r$. Hence this
  condition applies to every substring of $r$, which is what we nee.
\item $r$ is maximal w.r.t to Nate's defintion: Suppose this were not
  true: we could add 1 character to $r$  to form $r'$, and have an element that
  meats conditions Nate's definition.  (Note that, with no spaces, if
  we can add one or more characters we can add only 1 character -- we
  can assume are only adding 1.)  That means that every subsequence of
  $r'$ has the same frequency in $G$ (by condition (3) of Nate's
  defintion), meaning that they fufill condition (3) of our
  definition.  The other two conditions of our definition our
  immediate, meaning $r'$ contradicts (4) and hence $r$ is not an elementary
  repeat (by our definition).
\end{itemize}





\subsection{Lemmas}
Assume a fixed seed $s$ of length $l$ and a fixed genome $G$.

\begin{definition}
An $l$-mer {\it belongs} to a repeat element $r$ if the element is
compatable with an element in the maximal sequence decomposition of $r$.

\begin{lemma}
An $l$-mer can belong to at most one elementary repeat family.
\end{lemma}
Using our definition of ``belongs'', violating this would vilate
condition (3) of the r.e. definition.

\begin{observation}
For any string $l$-mer $w$, if $freq(\gamma_s(w)) \geq w$ than
$\gamma_s(w)$ is either is an elementary
repeat or is compatable with some substring of an elementary repeat.
\end{observation}
$\gamma_s(w)$ trivially satsfies the first three conditions, so only
the fourth condition could break it.

\begin{lemma}
Let $s_1$ and $s_2$ be two sequences in $G$ such that refd generalizes
both but there is no refd that generalizes generlizes superstrings
properly containing each, then $r$ is not compatable with a sub-refd
of any repeat element family.
\end{lemma}
Violation of this would imply a violation of (2) in the definition.

\begin{lemma}
If two $l$-mers belong to the same elementary repeat, then they have
the same frequency.
\end{lemma}
Given our definition of ``belongs'', this follow immediately.










\end{document}

%%  LocalWords:  REFDs mers
