\documentclass{article}
\usepackage{longtable}
\usepackage{ulem}
\pagestyle{myheadings}
\usepackage{ifthen}
\usepackage{color}

\setlength{\parindent}{0pt}
\setlength{\parskip}{1ex plus 0.5ex minus 0.2ex}

\newcommand{\red}[1]{{\color{red}#1}}
\newcommand{\blue}[1]{{\color{blue}#1}}
\newcommand{\cyan}[1]{{\color{cyan}#1}}
\newcommand{\magenta}[1]{{\color{magenta}#1}}

\newtheorem{definition}{Definition}
\newtheorem{observation}{Observation}
\newtheorem{lemma}{Lemma}

\begin{document}

\section{Definitions}

\subsection{REFDs}
In RAIDER, everything is defined in terms of $l$-mers.  Hence we need
to generlize slightly to allow the ambiguity character, so we will
create a description of a {\it repeat element family discriptior},
with an $l$-mer being a special case of that.

\begin{definition}
A \underline{repeat element family discriptor} (refd) is a sequence over
$\Sigma=\{A,C,G,T,*\}$ (where $*$ denotes unspecified bases).
\end{definition}

{\bf Note:} We will refer to sub-refds as consistent with the concept
of a substring, and denote then as consistent with Python.  (That is,
if $r =  AAAC*CGG$, then the sub-refd $r_{3:6}$, or $r[3:6]$, is the refd
$C*C$).  Generally, we will define all standard string operators on
refds unless otherwise noted.  (e.g. $\cdot$ will be the concatenation
operator, $|r|$ will denote total length, etc...)

\begin{definition}
Given an refd, $|r|$ denotes the length of the string and $||r||$
denotes the number of non-$*$ characters in the string.
\end{definition}
For $r = A*A$, $|r| = 3$ and $||r|| = 2$.

\begin{definition}
An $l$-mer is an refd of length $l$.
\end{definition}

\begin{definition}
Let $r$ and $r'$ be two $l$-mers.
\begin{itemize}
\item $r$ and $r'$ are \underline{consistent} if,
  for any position $i$ in which they differ, one of them contains a
  $*$.  ($AAC*TT$ and $AA*GTT$ are consistent; $AAC*TT$ and $AAGGTT$
  are not.)
\item We will say $r$ \underline{generalizes} $r'$ if, wherever they differ,
  $r$ has a $*$.  Denote this as $r \vdash r'$.  ($AAC*TT \vdash AACGTT$, 
  $AAC*TT \not\vdash AA*CTT$.)
\item The \underline{merge} of $r$ and $r'$, denoted $r \oplus r'$, is
  the $l$-mer such that $s_i = r_i$ when $r_i=r'_i$,
  and $*$ otherwise. ($AAC*TT \oplus AA*GTT = AA**TT$.) 
\item The \underline{tight merge} of $r$ and $r'$, denoted $r \otimes r'$,
  is $r \oplus r'$ if $r$ and $r'$ are consistent, and undefined
  otherwise. ($AAC*TT \otimes AA*GTT = AA**GT$, but $AAC*TT \otimes AAAA*CGT$ is undefined.)
\item The \underline{compression} of consistent refds $r$ and $r'$, denoted $r\ominus r'$, 
  is the string $s$ where $s_i$ is: $*$ if both $r_i$ and $r'_i$ are
  $*$, other the non-$*$ character at $r_i$ or $r'_i$.  If $r$ is and
  $r'$ are not consistent, $r \ominus r'$ is undefined.  ($AA**C
  \ominus A**CC = AA*CC$.  But $AA**C \ominus AT**C$ is undefined
  (because of the disagreement in the second position).
\end{itemize}
\end{definition}

{\bf Comments:}
\begin{itemize}
\item $r \oplus r'$ is the string that finds the $s$ over all
$l$-mers $s$ that generalizes both $r$ and $r'$ which maximizes $||s||$.
\item $r \ominus r'$ is the string that finds the $s$ over all
  $l$-mers $s$ that can be gerlized t both $r$ and $r'$ which
  maximizes $||s||$.
\item It might makes sense to say $r' < r$ if $r$ generlizes $r$ --
  which would then indce a lattice.  I have no idea if this is
  actually interesting / useful, and am ignoring it for now.
\end{itemize}


\begin{definition}
Let $x$ and $y$ be REFDs. Define $x \circ y$ as follows.  Let $i$ be
the largest $i$ such that $x_{|x|-i:}$ and $y_{:i}$ are consistent, let
$b = x_{|x|-i:} \ominus y_{:i}$, let $a = x_{:|x|-i}$ and $c = y_{i:}$.  
$x \circ y = a \cdot b \cdot c$.
\end{definition}
{\bf Examples:}
\begin{itemize}
\item $AAACCC \circ C*CTTT = AAACCCTTT$.
\item $AAACCC \circ CTCTTT = AAATCCTCTTT$.
\item $AAACC* \circ CC*TTT = AAACC*TTT$.
\end{itemize}


\subsection{REFDs and Spaced Seeds}


\begin{definition}
Given an $l$-mer $r$ and a spaced seed $s$ of length $l$, the seeded
$l$-mer, denoted $\sigma_s(r)$, is created by removing all bases at
positions corresponding to 0s.
\end{definition}
Example: $\sigma_{11011}(AACGG) = AAGG$. 

\begin{definition}
\label{generalize}
Given an $l$-mer $r$ and a spaced seed $s$ of length $l$, the
generlized $l$-mer (w.r.t. $s$), or $\gamma_s(r)$, is the refd created by replacing
all basses corresponding to 0s with * characters.
\end{definition}
Example: $\gamma_{110101}(AACGTG) = AA*G*G$.

\begin{definition}
Given a spaced seed $s$ of length $l$: 
\begin{itemize}
\item $s$ is {\it consistent} with an $l$-mer $r$ if, for all
$i$ such that $r_i = *$, $s_i = 0$.  
\item A spaced seed $s$ hits an refd $r$ at position $i$ if 
  $i <|r|-|s|$ and $s$ is consistent with $r_{i:i+|s|}$.
\end{itemize}
\end{definition}

\begin{definition} 
A spaced seed $s$ of length $l$ \underline{covers} and refd $r$ is, for every $j$,
there is an $i \leq j$ such that $s$ is consistent with the sub-refd
$r_{i:i+|l|}$.
\end{definition}
{\bf Examples:} Let $s = 11011$.
\begin{itemize}
\item $r = AAAAA*AA$. $s$ covers $r$ as we can place $s$ at $i=0$ and
  $i=3$ and, between them, cover every character in $r$ without
  concflict between a 1 and a $*$.
\item $r = AAA*AA$.  $s$ does not cover $r$.  $s$ only hits at $i=1$,
  so there is no way to cover base 0.
\item $r = AAAAAAAAAA$.  $s$ covers $r$, since $s$ hits at every base.
\end{itemize}

\subsection{REFDs and Sequence Sets}
Looks like we will need to define the relationship between refds and
sets of sequences to get the elementary repeat sequence.

Assume a fixed genome $G$.

\begin{definition}
Let $r$ be an refd of length $l$ and let $S$ be a set of sequences,
each of length $l$.
\begin{itemize}
\item We say $r$ is \underline{consistent} with $S$ if $r$ is
  consistent with every sequence in $S$.
\item We say $r$ is \underline{tightly consistent} with $S$ if $r$ is
  consistent with $s$ and, for every $i$ such that $r_i=*$, there
  exists sequences $x,y \in S$ such that $x_i \neq y_i$.
\end{itemize}
\end{definition}
That is: $r$ is tightly consistent with the set $S$ is $r$ only has
$*$ characters in those places it must.  (I thought this was going to
be necessary at one point, but am now doubting it.)

\subsection{REFD Decomposition}

Given an REDF $r$ and a seed $s$ of length $l$ ($l \leq |r|$), we want
to decompose $r$ into a set of overlapping $l$-mers.  These are the
$l$-mers that cannot be repeated outside of the repeat.

For example: if we have $G = \ldots AAAAACTTTTT\ldots AAAAAGTTTTT
\ldots$ (notice: 6 As), with the seed $1101$, we get the refd $AAAAAA*TT*TT$, which
deposes into $AA*AA$, $AA*TT$, and $TT*T$$.  If any string compatible
with any of these three occurs anywhere else, it breaks the family.
But if the string $AA*AC$ occurs somewhere else, it does not -- thats
irrelevant.  (In this example the minimal decomposition is unique -- I'm not
sure that this is always the case, or what multiple decompositions
mean.)

\begin{definition} 
Given a spaced seed $s$ of length $l$ and a an refd $r$ that is
covered by $s$, a \underline{decomposition} of $r$ w.r.t. $s$ is a set of coordinates $i$, $0 \leq i < |r|-l$, such
that (a) for each $i \in S$, $s$ hits $r$ at $i$, and (b) $r$ is
covered by $s$ using only those positions from $S$.
\end{definition}

Examples:
\begin{itemize}
\item $s=11011$, $r=AAAAA*CCCCC$, then the only decomposition is
  ${0,3,6}$.
\item $s=11011$, $r=AAAAAA*CCCCC$, then ${0,1,4,7}$ and ${0,4,7}$ are
  both decompositions.
\end{itemize}

\begin{definition}
A decomposition $S$ of $r$ with respect to $s$ is \underline{minimal}
if no element of $S$ can be removed without breaking the decomposition
properties. 
\end{definition}
In the above, the first set of the second example is not minimal. 

\begin{definition}
\label{rmdecomp}
Given an refd $r$ and a seed $s$ that covers $R4, the
\underline{right-most} decomposition, denoted $\delta_s(r)$ is the decomposition $S$ created
by:
\begin{enuemrate}
\item Set $i=0$.
\item Let $S = S \cup \{i\}$.
\item If $i == |r| - l$: quit.
\item Let $j = max_{i < j < i+|l|}\{s \mbox{ hits } r \mbox{ at } j$.
\item Goto (1).
\end{enumerate}
(In other words: Add $0$ to $S$, then keep taking then keep adding the
right-most index you can where $s$ hits $r$ and doesn't leave a gap
from the last hit.)

{\bf Note:} I think later we should consder allowing it to leave a
small gap to allow for insertions.

{\bf Examples:}
\begin{itemize}
\item $\delta_{11011}(AA*AATT*TT) = \{0,5\}$.
\item $\delta_{11011}(AA*AA*TT*TT) = \{0,3,6\}$.
\end{itemize}

\begin{observation}
A few observations about the decompositions of $r$ w.r.t. to seed $s$:
\begin{itemize}
\item Any decomposition must containt $0$ and $|r|-|s|$ (as these are
  the only way the first and last base can be covered).
\item $\delta_s(r)$ exists if and only if there is some decompostion of
  $r$ w.r.t $s$, which is true if and only if $s$ covers $r$.
\item $\delta_s(r)$ is unique if it exists.
\end{itemize}


\subsection{Elementary Repeats}

\begin{definition}
Let $s$ be a fixed seed with length $l$, $f$ be a fixed integer, and $G$ be a fixed
genomic sequence.  An
\underline{Elementary Repeat Family} is a tuple $F=(r,s,S)$ where: 
\begin{enumerate}
\item $s$ is a seed and $r$ is an refd such that $s$ covers $r$. 
\item $S$ is a set of genomic coordinates such that $|S| \geq f$.  Say
  that the sequences induced by $S$ are the substrings of $G$ starting
  at cordinates of $S$ and extended $|r|$ bases.
\item For every $i \in S$, $r$ is consistent with the string
  $G_{i:i+|r|}$.
\item Consider the sub-refds of $r$ used for $\delta_r(s)$ (definition
  \ref~{rmdecomp$}). None of these strings may generalize (definition
  \ref~{generalize}) any substring of $G$ outside ofthose induced by
  $S$.  (Minimiality condition.)
\item There is no refd $r'$ such that all of the following are true:
\begin{itemize}
  \item $r$ generalizes to some proper(?) substring of $v$ of $r'$.
  \item The strings in $\delta_r(s)$ make up part of the cover of
    $\delta_{s}(r')$.
  \item Every string induced in $G$ by $S$ is covered by a string that
    generalizes to $r'$.
\end{itemize}
\end{enumerate}
\end{definition}

{\bf Comments:} 
\begin{itemize}
\item I'm not sure if the maximality condition is what we want.
\item I haven't done anything to control the number of * symbols.
  Currently, we can have multipled refds for the same family by
  generalizing.  I'm not sure if this matters.
\end{itemize}


\begin{definition}
Given a repeat family $F=(r,S)$, the length of the family $|F|=|r|$,
and the size of the family $||F||=|S|$, the number of elements in the
family.
\end{definiton}


Alternative definition of elementary repeats:
\begin{definition}
Let $F=(r,s,S)$ be a repeat family with refd $r$, spaced seed $s$, and
coordinate set $S$ such that $s$ covers $r$. Let $C$ be the set of all
sub-refds corresponding to the right-most decompostion of $S$
w.r.t. $r$.  Then $F$ is a repeat element family (w.r.t. to $s$) if:
\begin{enumerate}
\item $|C| > 0$.
\item For each $i \in S$, $G[i:i+|r|]$ generalizes to $r$.
\item For each $w \in C$, the number of strings in $G$ that generlize
  to $w$ is $|S|$.  (Minimality -- but only applies to the
  decomposition.)
\item $F$ is maximal w.r.t to this definition.  Specifically, we
  cannot extend $r$ in either direction and have it covering exactly
  the same sequences.  (Need to think about this.)
\end{enumerate}




\subsection{Lemmas}

Right now, I'm trying trying to think of possibly interesting
statements and prove them.  Mostly just to get into the swing of this.

\begin{defintion} 
Conisider to elementary repeat families $F_1=(r_1,s,S_1)$ and
$F_2=(r_2,s,S_2)$.  These families overlap if there exist
some $i \in F_1$ and $j \in F_2$ such that either $0 \leq j-i < |r_1|$
or $0 \leq i-j < |r_2|$.
\end{definition}

That is, you have one sequence induced by each family such that they
actually overlap on $G$.


\begin{lemma}
Given repeat families $F_1$ and $F_2$ that overlap, and w.l.o.g. 
pick the $i$ and $j$ such that $0 \leq j-i < |r_1|$.  Then there exist
an offset vale $o = j-i$ such that, for any $i' \in S_1$, $j' \ in S_2$
s.t. $j' \geq i'$, either $j' - i' > |r_1|$ or $j' - i' = o$.
\end{lemma}
(In otherwords, corresponding sequences are offset by the same amount
every time.)



\begin{lemma} 
If repeat element families $F_1$ and $F_2$ overlap, then $||F_1|| \neq
||F_2||$.  
\end{lemma}









\end{document}
