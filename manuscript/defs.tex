\documentclass{article}
\usepackage{longtable}
\pagestyle{myheadings}
\usepackage{color}

\setlength{\parindent}{0pt}
\setlength{\parskip}{1ex plus 0.5ex minus 0.2ex}

\newtheorem{definition}{Definition}
\newtheorem{observation}{Observation}
\newtheorem{lemma}{Lemma}

\begin{document}

\section{Basic definitions}

My first attempt at providing a coherent mathematical definition of a
repeat element family in the context of a spaced seed.

\begin{definition}
  A {\it repeat element family descriptor (refd)} is a
  string over the alphabet $\{A,C,G,T,*\}$, describing the contents
  of any string in the family.
\end{definition}

\begin{definition}
  Given refds $r$ and $r'$, we will say $r'$ is a {\it slack
    substring} of $r$ (denoted $r' \prec r$) if there is a
  substring $r''$ of $r$ such that (1) $|r'|=|r''|$, and (2) for all
  $0 \leq i < |r'|$, either $r'_i = r''_i$, $r'_i = *$, or $r''_i=*$.
\end{definition}
(In otherwords, its a substring, with a potentially different *
pattern.)
{\bf Note: } I'm not sure if we should be allowing $r''_i=*$.

\begin{definition}
  We say an refd $r$ {\it matches} a Genome $G$ at position $i$ if, for
  all $j$ such that $r_j \neq *$, $r_j = G_{i+j}$.
\end{definition}

Example: If $r = AA*TT$, and $G = AACTTGGAAGTT$, then $r$ matches $G$
at positions $i=0$ and $i=7$.  If $G=AAATTT$, then $r$ matches $G$ at
positions $i=0$ and $i=1$.


\begin{definition}
  A spaced seed $s$ {\it matches} an refd $r$ at position $i$
  if, for every $0 \leq j < |s|$, $s_j = 0$
  whenever $r_{i+j}=*$.
\end{definition}

Example: If $s=11011$ and $r = AAAAA*GGG$, then $s$ matches $r$
at positions $i=0$ and $i=3$, but not at any other $i$.

\begin{definition}
  Given a seed $s$ and refd $r$, let $M_s(r)$ be the set of values $i$
  such that $s$ matches $r$ at $i$.
\end{definition}

Let $s=11011$ and $r = AA*CC*GGGGGG$.  Then $M_s(r) = [0, 3, 6, 7]$.


\begin{definition}
  A spaced seed $s$ is consistent with an refd $r$ if
  for every $i$, $0 \leq i < |r|$, there is some
  $i - |s| \leq j \leq i$ such that $s$ matches $r$ at position $j$.
\end{definition}

In other words: for any position $i$ of the refd, we must be able to
match the seed to a position of $r$ such that it then covers position
$i$.

Example: The seed $11011$ is compatible with $r=AA*AA*AA$.  (The seed
matches at positions 0 and 3, and all positions are covered by these
two.)  But it is not consistent with $AAAAA**AAAAA$, and there is no
seed that can match this string at any position that can cover $i=5$
or $i=6$.

\begin{observation}
  Let $L$ be the sortest sequence of the values in $M_s(r)$.  Then $s$
  is consistent with $r$ if and only if $\max_{0 < j < |s|-1} L[j+1]-L[j] \leq |s|$.
\end{observation}
That is, in the sorted list, every pair of adjacent elements must be
within $|s|$ of each other.

\begin{definition}
  Given a fixed genome $G$, and fixed spaced seed
  $s$, and a fixed value $f$, we define a elementary repeat family as
  a set $S$ of genome coordinates, $|S| \geq f$, such that there
  exists an refd $r$ where:
  \begin{itemize}
  \item $r$ matches the sequence of length $|r|$ starting at each
    element of $S$.  ($S$ is the set of all instances.)
  \item $s$ is consistent with $r$.  ($r$ corresponds to the seed.)
  \item There does not exist an refd $r'$, $r' \prec r$,  such that $r'$
    is consistent with $s$ and $M_s(r') - M_s(r) \neq \emptyset$.  (You
    cannot have a proper substring of $r$ that describes sequences
    outside of the instances described by $r$ -- minimality.)
  \item There does not exist an refd $r'$, $r \prec r'$, such that $s$ is consistent
    with $r'$, such that the instances defined by $M_s(r')$ contain all the
    instances of $M_s(r)$.  (Maximality.)
  \end{itemize}
\end{definition}
 {\bf Comment:} I'm not sure if the $r \prec r'$ is the right
 relationship.  Maybe just straight substring?  Or perhaps the
 definition of $\prec$ isn't quite right?


\begin{definition}
Given the refd $r$ of an elementary repeat family with set $S$, we say
that $r$ is {\it tight} if, for each $i$ such that $r_i = *$, there
exists two sequences defined by $S$ that have different bases in
position $i$.
\end{definition}
In other words: $r$ is tight if it only uses $*$ symbols where it must
to match everything sequence defined by $S$.

{\bf Comments:} I had it in mind that the algorithm would always
return a tight refd.  But in Carly's defense she gave the example:
s=11011, AAAAATCCCCC, AAGAATCCGCC, where ends up with AA*AA*AA*AA,
which is not tight.  Interestingly, if we have AAAAATTCCCCC and
AAGAATTCCGCC, then we get AA*AATTCC*CC -- to that extra T makes it
tight.  Not sure if this is significant.

\section{Algorithm}

Let $s$ be a fixed spaced seed, $G$ be a fixed genomic sequence, and
$f$ be a fixed integer $\geq 2$.  Let $f_s(i)$ be the sub-sequence at
$G[i]$ inducted by the seed (with 0-matching characters removed).  So $s = 11011$ and
the sequence at $G[i]$ is $AACGG$, the $f_s(i) = AAGG$.

Let $H$ and $F$ be hash-tables mapping strings to integers:
\begin{itemize}
\item $H[f(i)]$ maps reduced sequences to counts.
\item $F[f(i)]$ maps reduced sequences to family assignments.
  (e.g. $f(i)$ is assigned to family $F[f(i)]$.

Let $p(i)$ be the coorindate of $G$ the marks the beginning of the
last instance of $f(i)$.  That is, if $s=11011$ and $G =
ACCTCCAAACCTCC$, then $p(9) = 1$.  ($f(9) = CCCC$, and the largest
$i < 9$ such that $f_s(i) = CCCC$ is $i=1$.  (Not all functions are
pegged to a fixed $s$.)
 
\begin{verbatim}
num_fams <- 0 
for i <- 0 to |G|:
   seq = G[i:i+|s|]
   H[seq] += 1
   if H[seq] >= 2:
      pseq = 

\end{verbatim}

  
\section{Random Lemmas}

The following are things I wanted to try to prove that may or may not
be useful.






\end{document}
